% This is a template for students in MATH 3000 at FSU to use to keep track on various contributions.

% All of this stuff with '%' in front is a comment and ignored by the compiler.
%
% The lines before the "\begin{document}" line is called the preamble.
% This is where you load particular packages you need.
% Until you are more experienced, or the program says you are missing packages, it is safe to ignore most of the preamble.
%
%----------------------------------

\documentclass[12pt]{article}
\usepackage[margin=1in]{geometry}% Change the margins here if you wish.
\setlength{\parindent}{0pt} % This is the set the indent length for new paragraphs, change if you want.
\setlength{\parskip}{5pt} % This sets the distance between paragraphs, which will be used anytime you have a blank line in your LaTeX code.
\pagenumbering{gobble}% This means the page will not be numbered. You can comment it out if you like page numbers.

%------------------------------------

% These packages allow the most of the common "mathly things"
\usepackage{amsmath,amsthm,amssymb}

% This package allows you to add images.
\usepackage{graphicx}
\usepackage{float}

% Should you need any additional packages, you can load them here. If you've looked up something (like on DeTeXify), it should specify if you need a special package.  Just copy and paste what is below, and put the package name in the { }.  
\usepackage{wasysym} %this lets me make smiley faces :-)

% If you would like a more interesting title, go for it... this does feel a bit stiff and formal for our style.
\title{Geometry Curriculum Vit\ae}

% This is your CV, put your name here.
\author{Sarah Wright}

% This date should reflect the most recent update of this document.
\date{\today}

\begin{document}

\maketitle

\section*{Scribe}
% Anytime you act as a scribe for a peer's presentation and post the notes to Blackboard, add an entry to the list.  If a particular presentation required more to keep track of, you added a summary to you post, etc. please make a note of that.

\begin{enumerate}
\item Day \#1, 1/23, Marguerite's proof of Theorem 1.1(a).  This post includes a brief summary of the class discussion.
\item  
\end{enumerate}

% I can't think of any presentations that I've done, so these are blank for me... I should work on that.  There are two different types of presentations, both are important.  Include what you presented, the date, and if you collaborated with anyone else.
\section*{Progress Presentations}


\section*{Complete Presentations}


\section*{Papers}
% If you submit a formal write-up of a result, that goes here.  Give the title of the paper, the date of your original submission, and the date of publication.
\begin{enumerate}
\item  {\it Exterior Angle Theorem}; submitted 2/1; published 2/5
\item  
\end{enumerate}

\section*{Referee}
% If you referee a peer's paper, that goes here.  Include the title and author of the paper and the date(s) of your referee report(s).
\begin{enumerate}
\item  {\it Opposite Angles of Rhombus are Congruent}; Melissa Rheaume; 2/20
\item  {\it Construction of a Rhombus}; Shannon Jackson; 2/20
\item  {\it Rhombus Interior Triangle Congruence}; Dan Hanmore; 2/20
\item  
\end{enumerate}

\section*{Works in Progress}
% If there is a task you've been working hard on, alone or with others, but have not completed or given a progress presentation to the class, include that here. Include anyone you've worked with and a brief description of the status.
\begin{enumerate}
\item 
\end{enumerate}

\section*{Proposed Tasks}
% Use this to keep track of questions, conjectures, etc. that you have posed to the class.
\begin{enumerate}
\item  2/20-ish; Question L: What other properties guarantee that a quadrilateral is a rectangle? 
\end{enumerate}

\section*{Other}
% Use this (or create a new section with an appropriate name) for anything else where you spent a good chunk of time and effort but does not fit into any other category.

\end{document}