\documentclass[11pt]{article}
\usepackage{xltxtra}
\usepackage{bookmark}
\usepackage{hyperref}
\hypersetup{hidelinks}
\usepackage{url}
\urlstyle{tt}
\usepackage{multicol}
\usepackage{xcolor}
\usepackage{calc}
\usepackage{graphicx}
\usepackage{tikz}
\usetikzlibrary{calc}
\usepackage{fontspec}
\usepackage{xeCJK}
\usepackage{relsize}
\usepackage{xspace}
\usepackage{fontawesome}
\usepackage{titlesec}
\usepackage{enumitem}
\usepackage{siunitx}
\usepackage{amssymb}
\usepackage{tabularx}
\usepackage{multicol}
\usepackage{fontspec}

\CJKsetecglue{}							            % 取消中文字符与数字之间的间隔
\setlength{\parindent}{0pt}							% 取消全局段落缩进
\pagenumbering{gobble}								% 取消页码显示
%\setlist{noitemsep}									% 禁用列表中项目之间的额外垂直间距,但保留列表周围的间距
%\setlist{nosep}										% 禁用列表中项目之间的额外垂直间距及列表周围的间距
\setlist[itemize]{topsep=0em, leftmargin=*}		% 增加了itemize顶部间距
\setlist[enumerate]{topsep=0em, leftmargin=*}	% 增加了enumerate顶部间距

\titleformat{\section}					    % 将原标题前面的数字取消了
  {\LARGE\bfseries\raggedright} 		      % 字体改为LARGE,bold,左对齐
  {}{0em}                      			  % 可用于添加全局标题前缀
  {}                           			  % 可用于添加代码
  [{\color{NCU_Blue}\titlerule}]            % 标题下方加一条线
\titlespacing*{\section}{0cm}{*1.2}{*1.2}	% 标题左边留白,上方1.2倍,下方1.2倍

\titleformat{\subsection}				    % 将原二级标题前面的数字取消了
  {\large\bfseries\raggedright} 		      % 字体改为large,bold,左对齐
  {}{0em}                      			  % 可用于添加全局二级标题前缀
  {}                           			  % 可用于添加代码
  []
\titlespacing*{\subsection}{0cm}{*1.2}{*1.2}% 二级标题左边留白,上方1.2倍,下方1.2倍

% 页面大小与页边距,按需求调整
\usepackage[
	a4paper,
	left=1.2cm,
	right=1.2cm,
	top=1.5cm,
	bottom=1cm,
	nohead
]{geometry}

% 中文字符间距
\renewcommand{\CJKglue}{\hskip 0.05em}

% 英文字体
\setmainfont[
    Path=fonts/,
    Extension=.ttf,
    BoldFont=* Bold,
]{Microsoft Yahei}
% 中文字体
\setCJKmainfont[
    Path=fonts/,
    Extension=.ttf,
    BoldFont=* Bold,
]{Microsoft Yahei}

% 主题色
% 昌专蓝
\definecolor{NCU_Blue}{RGB}{29, 42, 84}

% 这里把表格的行间距调成1.2倍了
\renewcommand{\arraystretch}{1.2}
% 这里把正文的行间距调成1.2倍了
\linespread{1.2}

\newcommand{\school}{xxxxx学院 | School of xxxx}

% 联系方式
\newcommand{\contact}
{
    \small              % 换了更小的字号
    % \footnotesize       % 这比上面的小一号
    \scriptsize         % 这比上面的再小一号
    \textcolor{white}
    {
        \faEnvelope \quad \href{mailto:xxxx@xxx.com}{xxxx @xxx.com}    % 邮箱,前面的超链接可以直达邮箱软件
        \hspace{4em}    % 这里可以调间距
        \faPhone \quad xxxxxxxxxxxxx                % 手机号
    }
}

\begin{document}
	\begin{tikzpicture}[remember picture, overlay]
		\node[anchor=north, inner sep=0pt](header) at (current page.north){
			\includegraphics[width=\paperwidth]{images/background.png}
		};
		\node[anchor=west](school_logo) at (header.west){
			\hspace{0.5cm}
			\includegraphics[width=0.25\textwidth]{images/ncu_logo_2.png}
		};
		\node[anchor=east](school_name) at(header.east){
			\textcolor{white}{\textbf{\school}}
			\hspace{0.5cm}
		};
	\end{tikzpicture}
	\vspace{-4em}

	% 页脚,联系方式
	\begin{tikzpicture}[remember picture, overlay]
		\node[anchor=south, inner sep=0pt](footer) at (current page.south){
			\includegraphics[width=\paperwidth]{images/footer.png}
		};
        % 联系方式
        \node[anchor=center] at(footer.center){\contact};
	\end{tikzpicture}
	
	% 背景
	\begin{tikzpicture}[remember picture, overlay]
		\node[opacity=0.1] at(current page.center){
			\includegraphics[width=0.36\paperwidth, keepaspectratio]{images/ncu_logo_big.png}
		};
	\end{tikzpicture}

	% 个人信息
    \begin{figure}[h]
        % 左半边,信息,比例占行宽87%,可以自己调
        \begin{minipage}{0.83\textwidth}
            \section{\makebox[\widthof{\faUser}][c]{\color{NCU_Blue}{\faUser}}\quad 个人信息}
            \begin{tabularx}{\linewidth}{p{\widthof{出生日期:}}Xp{\widthof{政治面貌:}}X}
                姓\qquad 名: & xxx & 性\qquad 别: & x \\
                出生日期: & 2003年xx月xx日 & 政治面貌: & xxxx \\
                联系方式: & xxxxxxxxxxxx & 电子邮箱: & \href{mailto:xxxx@xxx.com}{xxxx @xxx.com} \\
            \end{tabularx}
        \end{minipage}
        % 右半边,照片,比例占行宽12%,可以自己调
        % images/example_avatar.png 替换成你证件照的路径。
        \begin{minipage}{0.16\textwidth}
            \includegraphics[width=\linewidth]{images/avatar.png}
        \end{minipage}
        % 尽量留至少1%的间距,不然会换行
    \end{figure}


	
	\section{\makebox[\widthof{\faGraduationCap}][c]{\color{NCU_Blue}{\faGraduationCap}}\quad 教育背景}
	\vspace{-1em}
    \begin{table}[h!]
        \begin{tabularx}{\textwidth}{XXp{\widthof{2021年 -- 预计2025年7月毕业}}}
            xxxx学院 & xx学 & 2021年 -- 预计2025年7月毕业\\
            \textbf{GPA: x.xx/4} & \textbf{GPA排名: x/xx} \\%& \textbf{综测排名: ??/??} \\
        \end{tabularx}
    \end{table}

    % 所获荣誉(这个看你想不想写了)
    \section{\makebox[\widthof{\faStar}][c]{\color{NCU_Blue}{\faStar}}\quad 所获荣誉}
    \vspace{-1em}
    \begin{multicols}{2}
        \begin{itemize}
            \item 2021-2022xxxx
        \end{itemize}
    \end{multicols}

    % 竞赛经历
    \section{\makebox[\widthof{\faTrophy}][c]{\color{NCU_Blue}{\faTrophy}}\quad 竞赛经历}
    \vspace{-1em}
    \begin{table}[h!]
        \begin{tabularx}{\textwidth}{Xp{\widthof{第零负责人}}p{\widthof{国家级-第100名}}p{\widthof{2030年13月}}}
            \textbf{20xxxxxxxxx比赛} & 第x负责人 & x级-x等奖 & 20xx年x月 \\
            % 同理,可以自己加
        \end{tabularx}
    \end{table}

    % 项目经历
    \section{\makebox[\widthof{\faGears}][c]{\color{NCU_Blue}{\faGears}}\quad 项目经历}
    \vspace{0.5em}
    
    \subsection{xxxx \hfill 科研训练项目}
    \hfill 20xx年xx月-20xx年xx月
    
    xxx
    
    

    \section{\makebox[\widthof{\faWrench}][c]{\color{NCU_Blue}{\faWrench}}\quad 技能特长}
    \vspace{0.5em}
    \begin{itemize}
        \item 熟练使用xxx。
    \end{itemize}

\end{document}
