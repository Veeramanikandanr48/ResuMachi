\documentclass{article}
\usepackage[utf8]{inputenc}
\usepackage[left=1in,top=0.6in,right=1in,bottom=0.6in]{geometry}
\usepackage{cmbright}
\usepackage{xcolor}

\title{my-resume-template}
\author{briankamras }
\date{July 2019}

\pagenumbering{gobble}

%-----Defining Accent Colors-----

\definecolor{cream}{HTML}{eff0f1}
\definecolor{light-blueish}{HTML}{3daee9}
\definecolor{brey}{HTML}{222222}
\definecolor{cyan}{HTML}{005577}

%\pagecolor{cream} %-----Use if you want a resume slightly easier to look at on digital devices-----

\newcommand\liner{\leavevmode\xleaders\hbox{\textcolor{cyan}{--}}\hfill\kern0pt}

\begin{document}

%-----Header Area-----

\begin{center}
\begin{tabular}{p{5cm}|p{10cm}}
  \large{\textbf{Brian} Leon Kamras, PhD email@address.com (555) 555-5555} & \large{Creative Chemist with 6 years experience in research and development (nano-materials, polymers, colloids and metals), product formulations and project management.  Skilled in development of polymer hybrids for injectable medical devices, wound dressings, and diagnostic tools.  Interested in material design, new product development, and basic science.}\\
\end{tabular}
\end{center}

\vspace{12pt}

%-----Skills-----

\large{\textcolor{brey}{SKILLS}} \liner

\begin{table}[h]
  \begin{center}
    \begin{tabular}{cc}
       Nanomaterials Synthesis/Characterization & C, Python, {\LaTeX}\\
       Medicinal Chemistry & Experimental Design \\
       Polymer Chemistry & Fluorescent and Phosphorescent Spectroscopy \\
       Inorganic Chemistry & MS Office, Origin, Linux, UNIX \\
       Surface Chemistry & Leadership/Management (Undergraduate, Graduate) \\
       UV-Vis, FT-IR, NMR, Raman & HPLC, GC, GC-MS \\
       Aseptic Technique & Technical Writing (SOPs, Protocols, Papers) \\
       Cell Culture (Maintenance, Assays) & Conversational German (Native English) \\
    \end{tabular}
  \end{center}
\end{table}

%-----Experience-----

\large{\textcolor{brey}{WORK EXPERIENCE}} \liner

\vspace{10pt}

\normalsize{
\textbf{Gold Nanotriangles Using Non-toxic Chitosan} \hfill 2013 - 2014
\begin{itemize}
\item Used Response Surface Methodology and empirical analysis to optimize synthethis of near-infrared (NIR) absorbing gold nanoparticles (NIRNPs) \& nanotriangles
\item Used modified Job plots, UV-Vis-NIR spectroscopy and SEM to determine that chitosan concentration affects particle size and shape
\item \textit{Result: NIRNPs used as photothermal agents for variety of studies in Omary lab, initial patent expanded}
\end{itemize}

\textbf{Gold Nanorods Using Lecithin} \hfill 2014 - 2016
\begin{itemize}
\item Used crude egg lecithin as "drop in" replacement for CTAB
\item Synthesized size-tunable gold nanoparticle seeds and gold nanorods with this method
\item \textit{Result: Patent initiated and resulting lecithin+nanoparticle conjugate investigated as multimodal theranostic platform}
\end{itemize}

\textbf{Size Tunability of Polymer Nanoparticles} \hfill 2016 - 2018
\begin{itemize}
\item Developed block copolymer nanoparticles for biomedical applications
\item Used non-toxic stabilizer to create particles with nanometer-precise diameter
\item Developed new mathematical relationship between reagents and particle size explaining influence of surfactnat on size
\item \textit{Result: nanoparticles and synthetic method used as platform for biomedical studies in Omary Lab}
\end{itemize}

\textbf{NP@FMOF-1 Nanomaterials as Room-temperature Catalysts} \hfill 2018 - 2019
\begin{itemize}
\item Grew metal nanoparticles within metal-organic-framework for carbonation of value-added feedstocks
\item Characterized crystal properties and catalytic yields using NMR, PXRD, and SAXS
\end{itemize}
}

\vspace{10pt}

\large{\textcolor{brey}{TEACHING AND MENTORSHIP EXPERIENCE}} \liner

\normalsize{
\begin{itemize}
  \item Undergraduate Labs (General Chemistry (1 \& 2) and Organic Chemistry (1 \& 2)
  \item Organic Chemistry lecturing
  \item Mentorship of graduate students on instrument use, experimental technique, and presentation skills
\end{itemize}
}

\vspace{10pt}

%-----Education-----

\large{\textcolor{brey}{EDUCATION}} \liner

\textbf{Doctor of Philosophy, University of North Texas} \hfill August 2013 - May 2019\\

\textbf{Bachelor of Arts, Austin College} \hfill August 2009 - May 2012\\


%-----Selected Awards-----

\large{\textcolor{brey}{Selected Awards}} \liner

\vspace{10pt}

\normalsize{
\textbf{Brookhaven National Labs - Center For Functional Nanomaterials} \hfill September 2019\\
\begin{itemize}
  \item[] Awarded instrument use based on competitive proposal: “Investigation of surface chemisty, crystal structure, and composition of a nanoparticle-embedded fluorous metal organic framework (FMOF-1)”
\end{itemize}

%-----Selected Pubs-----

\large{\textcolor{brey}{PUBLICATIONS}} \liner

\vspace{10pt}
\small{
\begin{itemize}
\item B. L. Kamras.  "Application-focused Investigation of Monovalent Metal Complexes for Nanoparticle Synthesis." 2019.
\item D. K. Korir, B. Gwalani, A. Joseph, B. Kamras, R. K. Arvapally, M. A. Omary and S. B. Marpu, Nanomaterials, 2019, 9, 596.
\item B. L. Kamras, N. M. Nasiri, D. Korir, D. P. Simmons, M. A. Omary, Journal of Physical Chemistry C, 2019, in progress.
\end{itemize}
}

\large{\textcolor{brey}{SELECTED CONFERENCES}} \liner

\vspace{10pt}

\small{
\begin{itemize}
  \item Brian L. Kamras and Mohammad A. Omary. “Environmentally Benign, Biocompatible Gold Nanoparticles for Photothermal Therapy”.  National Cancer Institute Center for Strategic Science Initiatives (NCI-CSSI), University of North Texas Discovery Park,  August 2014.
  \item Nooshin M. Nasiri, Brian Leon Kamras, Sreekar Marpu, Denise Perry Simmons, Mohammad A. Omary. “New Synthesis Methodology for Making FITC Labeled PMMA Nanoparticles: Understanding Effect of Crosslinked vs. Surfactant-stabilized  Nanoparticles on Conjugation”. ACS National Meeting, San Diego, 2019.
\end{itemize}
}

\end{document}
